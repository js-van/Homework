\documentclass{article}
\usepackage{amsmath}

\begin{document}

\paragraph{1.1}
\subparagraph{a} Let $I(Y) = \left \langle x^2 - y \right \rangle$ and $f,g \in A^2$.  Construct a mapping $\phi : A^2 / I(Y) \rightarrow k[t]$ such that:
\[ \phi ( f(x, y) ) = f(t, t^2) \]
$\phi$ is a ring homomorphism since $\phi( (f g)(x,y) ) = (f g)(t, t^2) = f(t,t^2) g(t,t^2)$ and $\phi((f + g)(x,y)) = f(t,t^2) + g(t,t^2)$.  $\phi$ is also injective:
\begin{eqnarray*}
\phi( f(x,y) + g(x,y) (x^2 - y) ) & = & \phi(f(x,y)) + g(t,t^2) (t^2 - t^2) \\
	& = & \phi(f(x,y))
\end{eqnarray*}
And so $\phi(f) = \phi(g)$ iff $f = g$.  Finally, $\phi$ is surjective since for all $p \in k[t]$, there exists some $f(x,y) = p(x) \in A^2$ such that $\phi(f) = p$.  Therefore $\phi$ is an isomorphism and $A(Y) \cong k[t]$.

\subparagraph{b}
The equivalence classes of $A(Z)$ are isomorphic to $A^1 \dot{\cup} A^1$.  Topologically, this consists of two disconnected components and so it cannot be isomorphic to $A^1$

\subparagraph{c}
Take any quadratic $a x^2 + b x y + c y ^2 \in A$ (it suffices to consider this general form, since the linear component could be removed via translation/scaling of the curve).  Factor the expression via the quadratic formula to get an expression for $x$:
\[ x = \frac{-b \pm \sqrt{b^2 - 4 a c} } { 2 a } y \]
If $b^2 - 4 a c =  0$, then the expression is single valued, and so via the substitution trick from part a we conclude that the coordinate ring for the conic is identically $k[x]$.  Otherwise, the expression is two-valued and so looks like $A(Z)$

\paragraph{1.2}
First $Y = \{ y - x^2 = 0 \} \cap \{ z - x^3 = 0 \}$, which follows from placing $x$ in bijection with $t$.  Consequently, $Y$ is the intersection of two varieties; $\{ y - x^2 = 0 \}$ and $\{ z - x^3 = 0 \}$ and so $I(Y) = \left \langle y - x^2, z - x^3 \right \rangle $.  Because $I(Y)$ is principle and has 2 generators, $\textrm{height }I(Y) = 2$ and so:
\[ \textrm{dim } Y = \textrm{dim }A^3 - \textrm{height }I(Y) = 1 \]
Finally picking $\phi( f(x,y,z) ) = f(t, t^2, t^3)$ gives an isomorphism by an argument symmetric to 1.1a.

\paragraph{1.3}
Consider the ideals $I(Y_1) = \left \langle z - 1, x^2 - yz \right \rangle, I(Y_2) = \left \langle x, y \right \rangle, I(Y_3) = \left \langle x, z \right \rangle$.  Direct calculation shows that $I(Y) = \left \langle x^2 - yz, xz -x \right \rangle = I(Y_1) \cap I(Y_2) \cap I(Y_3)$.  Moreover, since each $I(Y_i)$ is generated by irreducible polynomials, the varieties $Y_1, Y_2, Y_3$ are irreducible.

\paragraph{1.7}
\subparagraph{a}
That $ii \Rightarrow i$ is obvious (since a sequence of sets is also a family).  To show $i \Rightarrow ii$, let $S$ be a family of closed subsets of $X$.  Because $X$ is Noetherian, any sequence $s_i \supseteq s_{i+1} \supseteq ... $ of closed subsets in $X$ has a minimal element.  Since $S$ forms a partial ordering by inclusion, Zorn's lemma states that $S$ has a minimal element. 

Now, to show that $i \Rightarrow iii$, consider any sequence of open sets $t_i \subseteq t_{i+1} \subseteq ...$ with $t_i \subseteq X$.  Taking the complement of each $t_i$ gives a collection of closed sets $\bar{t_i}$ such that $\bar{t_i} \supseteq \bar{t_{i+1}} \supseteq ... $, which by $i$ contains a minimal element, $\bar{t_n}$.  Therefore, $t_n$ is the maximal element of $t_i, t_{i+1}, ...$.

A symmetric argument applies to show that $iii \Leftrightarrow iv$ and $iii \Rightarrow i$ (replacing all instances of open with closed, and $\supset$ with $\subset$).

\subparagraph{c}
Since the closed sets in a subset of $X$ are contained in $X$; a closed sequence of such subsets is alos a closed sequence in $X$ and so it must have a minimal element and thus all subsets of $X$ are also Noetherian.

\paragraph{1.8}
Split $H$ into irreducible components and consider each individually.  If we take the union of a height $n - r$ ideal with a height $1$ prime ideal, the resulting ideal is height $n - r - 1$ unless the height 1 ideal is contained in the $n-r$ ideal.  Therefore, the dimension of each irreducible component of the intersection must be $r - 1$ (unless the intersection is strictly contained in H).

\paragraph{1.9}
If $a$ is prime, then $\textrm{height }a = r$ and so the equality is satisified.  Otherwise, $\textrm{height } a < r$.  Therefore;
\[ \textrm{dim } A = \textrm{dim } A^n - \textrm{height } a \geq n - r \]

\paragraph{1.11}
$I(Y) = \{ x^4 - y^3, y^5 - z^4, x^5 - z^3 \}$

\end{document}
