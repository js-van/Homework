\documentclass{article}
\usepackage{amsmath,amsfonts}

\begin{document}

\paragraph{2.4}
\subparagraph{a}
\emph{WTS:} There is a 1-1 inclusion reversing correspondence between algebraic sets in $\mathbb{P}^n$ and homogeneous radical ideals of $S$ not equal to $S_+ = \bigoplus_{d>0} S_d$ given by:
\[ Y \subseteq \mathbb{P}^n \mapsto I(Y) = \{ f \in S | f \textrm{ is homogeneous and } f(P) = 0 \textrm{ for all } P \in Y \} \]
\[ \mathfrak{a} \in S \mapsto Z(\mathfrak{a}) = \{ P \in \mathbb{P}^n | f(P) = 0 \forall f \in \mathfrak{a} \} \]

\emph{WTS:} $Y = Z(I(Y))$

By definition $Y \subseteq Z(I(Y))$. Because $Y$ is algebraic, there exists some $f \in S$ such that $f(P) = 0 \Leftrightarrow P \in Y$.  As a result, $f \in I(Y)$ and so $Z(I(Y)) \subseteq Y$.  Therefore we have an equality.

\emph{WTS:} $\mathfrak{a} = I(Z(\mathfrak{a}))$

Once again, it is obvious that $\mathfrak{a} \subseteq I(Z(\mathfrak{a}))$.  Moreover, because $\mathfrak{a}$ is radical, by the Nullstellensatz $I(Z(\mathfrak{a})) \subseteq \mathfrak{a}$ and so the correspondence is 1-1.

\emph{WTS:} For all varieties $Y_1, Y_2 \subseteq \mathbb{P}^n$:
\[ Y_1 \subset Y_2 \implies I(Y_1) \supset I(Y_2) \]

This follows because if $Y_1 \subset Y_2$, then $f(P) = 0$ for all $P \in Y_2$, then $f(Q) = 0$ for all $Q \in Y_1$ and so $f \in I(Y_1)$.

\emph{WTS:} For all radical ideals $\mathfrak{a}_1, \mathfrak{a}_2 \in S \setminus \{ S_+ \}$:
\[ \mathfrak{a}_1 \subset \mathfrak{a}_2 \implies Z(\mathfrak{a}_1) \supset Z(\mathfrak{a}_2) \]

If $f \in \mathfrak{a_2}$, $f(Q) = 0 \implies Q \in Z(\mathfrak{a_1})$ and so $Q \in Z(\mathfrak{a}_1) \implies Q \in Z(\mathfrak{a}_2)$.  Therefore the correspondence is inclusion reversing.  $\Box$

\subparagraph{b}
An algebraic set $Y \subseteq \mathbb{P}^n$ is irreducible if and only if $I(Y)$ is a prime ideal.

\emph{WTS:} $Y$ irreducible $\implies$ $I(Y)$ is prime.

If $Y = \emptyset$, then $I(Y)$ contains only constant functions and so it is prime.

$Y \neq \emptyset$ is irreducible if it cannot be expressed as the union $Y = Y_1 \cup Y_2$ of two proper closed subsets.

Suppose $I(Y)$ is not prime; then there exists a pair $f,g \in S \setminus I(Y)$ such that $fg \in I(Y)$, then $Y \subseteq Z(fg) = Z(f) \cup Z(g)$, thus
\[ Y = (Y \cap Z(f)) \cup (Y \cap Z(g)) \]
both of which are closed.  But since $Y$ is irreducible, either $Y \subseteq Z(f)$ or $Y \subseteq Z(g)$ and so one of them must be in $I(Y)$.  But this is a contradiction and so we must conclude that $I(Y)$ is prime.

\emph{WTS:} $\mathfrak{a}$ prime $\implies Z(\mathfrak{a})$ irreducible.

Suppose that $Z(\mathfrak{a}) = Y_1 \cup Y_2$; then $I(Z(\mathfrak{a})) = \mathfrak{a} = I(Y_1) \cap I(Y_2)$.  But $\mathfrak{a}$ is prime so either $\mathfrak{a} = I(Y_1)$ or $\mathfrak{a} = I(Y_2)$.  Therefore $Z(\mathfrak{a}) = Y_1$ or $Y_2$, and hence it is irreducible.

$\Box$

\subparagraph{c}
$\mathbb{P}^n$ is irreducible.

By part b, it is enough to show that $I(\mathbb{P}^n)$ is prime.  However, $I(\mathbb{P}^n) = \{ 0 \}$, which is trivially prime.  Therefore $\mathbb{P}^n$ is irreducible. $\Box$

\paragraph{2.8}
\emph{WTS:} A projective variety $Y \subseteq \mathbb{P}^n$ has dimension $n - 1$ if and only if it is the zero set of a single irreducible homogeneous polynomial $f$ of positive degree.

From 2.6, we know that $\textrm{dim } S(Y) = \textrm{dim } Y + 1$ and by 2.7 we know that $\textrm{dim } \mathbb{P}^n = n$.  Since $S(Y)$ is homogeneous and $I(Y)$ is prime (given that $Y$ is a variety), proposition 1.7 implies that

\begin{eqnarray*}
\textrm{dim } Y & = & \textrm{dim } \mathbb{P}^n - \textrm{height } I(Y) \\
 & = & n - 1
\end{eqnarray*}

$\Box$
\paragraph{2.9}
If $Y \subseteq \mathbb{A}^n$ is an affine variety, we identify $\mathbb{A}^n$ with an openset $U_0 = \mathbb{P}^n \setminus \{ x_0 = 0 \}$ by the homeomorphism $\varphi_0 : U_0 \to \mathbb{A}^n$ where 
\[ \varphi_0(x_0, ..., x_n) = \left ( \frac{x_1}{x_0}, ..., \frac{x_n}{x_0} \right ) \]
Then define the projective closure $\bar{Y}$ such that:
\[ \bar{Y} = \bigcap \{ Y' \subseteq \mathbb{P}^n | Y' \textrm{ is algebraic and } Y \subseteq \varphi_0(Y') \} \]

\subparagraph{a}
\emph{WTS:} $I(\bar{Y})$ is the ideal generated by $\beta(I(Y))$, where
\[ \beta(g) = x_0^e g \left ( \frac{x_1}{x_0}, ... , \frac{x_n}{x_0} \right ) \]
and $e = \textrm{deg}(g)$.  Recall:

\[ I(\bar{Y}) = \{ f \in S | f(p) = 0 \textrm{ for all } p \in U_0 \wedge \varphi_0(p) \in Y \} \]

\[ \beta(I(Y)) =  \{ \beta(g) | g \in A \wedge g(x) = 0 \textrm{ for all } x \in Y \} \]

For all $g \in I(Y)$, there exists a $f \in S$ such that $f \circ \varphi_0 = \beta \circ g$; and so taking the closure of $Z(\beta(I(Y) ))$ corresponds to finding the ideal generated by $\beta(I(Y))$.  As a result $I(\bar(Y))$ is generated by $\beta(I(Y))$. $\Box$

\subparagraph{b}
Let $Y = \{ (t, t^2, t^3) | t \in k \}$.

From homework 1, $I(Y)$ is generated by $\{ y - x^2, z - x^3 \}$.

However, $I(\bar{Y}) = \langle y w - x^2, x z - y^2,  x w - y z \rangle$, which is generated by elements homogeneous of degree 2.  As a result, applying $\beta$ to the generators of $I(Y)$ does not always yield generators for $I(\bar{Y})$. $\Box$

\paragraph{2.12}
For given $n,d > 0$ let $M_0, M_1,... M_N$ be all the monomials of degree $d$ in $n+1$ variables $x_0,..,x_n$ where $N = {n+d \choose n} - 1$.  Define a mapping $\rho_d : \mathbb{P}^n \to \mathbb{P}^N$ by sending $P = (a_0, ... , a_n)$ to the point $\rho_d(P) = (M_0(a), M_1(a), ..., M_N(a))$ obtained by substituting $a_i$ in the monomials $M_j$.

\subparagraph{a}
Let $\theta : k[y_0, ... y_N] \to k[x_0, ..., x_n]$ be the homomorphism defined by sending $y_i$ to $M_i$ and let $\mathfrak{a}$ be the kernel of $\theta$. 

\emph{WTS:} $\mathfrak{a}$ is a homogeneous prime ideal.

The fact that $\mathfrak{a}$ is an ideal follows from the fact that it is a kernel.  Moreover, $\textrm{Im } \theta$ is a polynomial ring over an algebraically closed field and so $k[x_0, ... , x_n] / \mathfrak{a}$ is entire and thus $\mathfrak{a}$ is prime.  

That $\mathfrak{a}$ is homogeneous follows from the fact that each $M_i$ is a homogeneous monomial of degree $d$, and thus substituting variables of homogeneous degree does not change the homogeneity of a polynomial.

The fact that $Z(\mathfrak{a})$ is a variety follows from 2.4(b) and the above predicate. $\Box$

\subparagraph{b}

\emph{WTS:} $Z(\mathfrak{a}) \subseteq \rho_d(\mathbb{P}^n)$

\emph{WTS:} $Z(\mathfrak{a}) \supseteq \rho_d(\mathbb{P}^n)$



\subparagraph{c}

\emph{WTS:} $\rho_d$ is a homeomorphism of $\mathbb{P}^n$ onto $Z(A)$.

From part b, we know $\rho_d$ is onto.  Moreover, since $\rho_d$ is a polynomial it sends closed sets to closed sets; so all that remains is to check that:

\emph{WTS:} $\rho_d$ is 1-1.

Suppose there exist $p, q \in \mathbb{P}^n$ such that $p \neq q$ and $\rho_d(p) = \rho_d(q)$.  However, since $p \neq q$, there must exist some term $x_j$ such that $p_j \neq q_j$.  Now consider the monomial term $M_i = x_j^d$.  Clearly $\rho_d(p)_i \neq \rho_d(q)_i$ and so we have a contradiction.  $\Box$

\subparagraph{d}
Pick $n = 1, d = 3$, then $N = {n + d \choose n} - 1 = 3$ and 
\[ \rho_3(x_0, x_1) = (x_0^3, x_0^2 x_1, x_0 x_1^2, x_1^3 ) \]
\[ I(\rho_3 (\mathbb{P^1})) \cong \langle y w - x^2, x z - y^2,  x w - y z \rangle \] $\Box$

\paragraph{2.14}
Let $\psi : \mathbb{P}^r \times \mathbb{P}^s \to \mathbb{P}^N$ be defined by sending the ordered pair $(a_0, ..., a_r) \times (b_0, ..., b_s)$ to $(..., a_i b_j, ...)$ lexicographically, where $N = rs + r + s$.  It is evident that $\psi$ is well-defined and injective.

\emph{WTS:} $\textrm{Im } \psi$ is a subvariety of $\mathbb{P}^N$.

Consider the ring homomorphism $\phi : k[z_{i,j}] \to k[x_0, ... , x_r, y_0, ... , y_s]$ such that $\phi(z_{i,j}) = x_i y_j$.  The kernel of this map is clearly a prime ideal (since its image is a polynomial ring).  Likewise $\phi$ is the dual of $\psi$ acting on homomorphism, and so $\textrm{Ker } \psi \cong I(\textrm{Im }\phi)$ or $Z(\textrm{Ker }\phi) = \textrm{Im } \psi$. $\Box$

\paragraph{2.15}
Define $Q \subseteq \mathbb{P}^3$ by the equation $xy - zw = 0$.

\subparagraph{a}
\emph{WTS:} $Q = \psi( \mathbb{P}^1 \times \mathbb{P}^1 )$ where
\[ \psi(s, t, u, v) = (s u, t v, t u, s v) \]
Picking coordinates $x,y,z,w$ gives the equations:
\begin{eqnarray*}
x & = & s u \\
y & = & t v \\
z & = & t u \\
w & = & s v 
\end{eqnarray*}
Solving for $s,t,u,v$ in terms of $x,y,z,w$ gives:
\begin{eqnarray*}
s & = & \frac{x}{u} \\
u & = & \frac{z}{t} \\
t & = & \frac{y}{v} \\
v & = & \frac{w}{s}
\end{eqnarray*}
Substituting into $s$:
\[ s = \frac{x}{u} = \frac{x t}{z} = \frac{x y}{z v} = \frac{x y s}{z w} \]
Cancelling $s$ gives $1 = \frac{xy}{zw}$ or $xy - zw = 0$ which is exactly $Q$. $\Box$

\subparagraph{b}

\emph{WTS:} $Q$ contains two families of lines, $\{ L_t \}, \{ M_t \}$ parameterized by $t \in \mathbb{P}^1$ such that for all $t,u \in \mathbb{P}^1$; $L_t \neq L_u \implies L_t \cap L_u = \emptyset$; $M_t \neq M_u \implies M_t \cap M_u = \emptyset$ and $L_t \cap M_u = $ one point.

Because $\mathbb{P}^1$ is projective, we split the parameter $u$ into a pair $(s,t)$ modulo the relation $(s,t) \cong (\lambda s, \lambda t)$. Pick $L_u = \{ (x,y,z,w) \in Q | sx - tz \}$ and $M_u = \{ (x,y,z,w) \in Q | sy - tw \}$; which clearly satisfy the projective equivalence relation.  

For any pair of points $p,q \in \mathbb{P}^1$, where $p = (s,t), q = (u,v)$; $p \neq q \implies L_p \neq L_q$. Moreover the intersection term is given by:  
\begin{eqnarray*}
sx - tz & = & 0 \\
ux - vz & = & 0 \\
xy - zw & = & 0
\end{eqnarray*}
However, this system is overdetermined if $u/v \neq s/t$, and so the only possible choice for $xy$ is 0.  But this is not in $\mathbb{P}^3$ and therefore $L_p \cap L_q = \emptyset$.  Symmetrically $M_p \neq M_q \implies M_p \cap M_q = \emptyset$.  Now consider $L_p \cap M_q$.  This gives the system of equations:
\begin{eqnarray*}
sx - tz & = & 0 \\
uy - vw & = & 0 \\
xy - zw & = & 0
\end{eqnarray*}
The solution to this system is the set given by the set $(x,y,z,w) = (t, w, s, u)$ which is a point $\mathbb{P}^3$ and so $\L_p \cap M_q$ is in fact a point. $\Box$

\subparagraph{c}

\emph{WTS:} $Q$ contains curves not contained in $M_t \cup L_t$: ie the twisted cubic:
\[ c = \{ (x,y,z,w) \in \mathbb{P}^3 | yw - x^2 = 0, xz - w^2 = 0, xy - zw = 0 \} \]

However the only curves contained in $\mathbb{P}^1 \times \mathbb{P}^1$ are the families $L_p$ and $M_q$ as described above.  Yet, $c$ intersects each curve in $L_p$ and $M_q$ such that the region $c \cap L_p$ given by:
\begin{eqnarray*}
sx - tz & = & 0 \\
yw - x^2 & = & 0 \\
xz - w^2 & = & 0 \\
xy - zw & = & 0
\end{eqnarray*}
Substituting $x = tz / s$:
\begin{eqnarray*}
yw - \frac{t^2}{s^2}z^2 & = & 0\\
\frac{t}{s}z^2 - w^2 & = & 0 \\
\frac{t}{s}zy - zw & = & 0
\end{eqnarray*}
And so $w = \pm \sqrt{\frac{t}{s}} z$:
\begin{eqnarray*}
\pm \sqrt{\frac{t}{s}} yz  - \frac{t^2}{s^2} z^2 & = & 0 \\
\frac{t}{s}y z \mp \sqrt{\frac{t}{s}} z^2 & = & 0 
\end{eqnarray*}
Which has a solution for $\frac{t}{s} = 0, 1$.  However by part b, $c$ can not be in either $\{ L_p \}$ or $\{ M_p \}$ since it intersects two curves in both sets.  Therefore $c \not \in \mathbb{P}^1 \times \mathbb{P}^1$ and so we must conclude that the Zariski topology on $Q$ is distinct from $\mathbb{P}^1 \times \mathbb{P}^1$. $\Box$

\end{document}
