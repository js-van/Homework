\documentclass{article}
\usepackage{amsmath,amsfonts,graphicx,fullpage,comment}

\newcommand{\HRule}{\rule{\linewidth}{0.5mm}}
\begin{document}

\begin{titlepage}
 
\begin{center}
 
\textsc{\LARGE Homework 2}\\[1.5cm] 

\textsc{\Large Computational Math}\\[0.5cm]
 
 
\HRule \\[2cm]
 
% Author and supervisor
\begin{minipage}{0.4\textwidth}
\begin{flushleft} \large
\emph{Author:}\\
Mikola Lysenko
\end{flushleft}
\end{minipage}
 
\vfill
 
% Bottom of the page
{\large \today}
 
\end{center}
 
\end{titlepage}


\paragraph{1}

\subparagraph{a}

I checked the exactness condition for the given $u$ using sympy and a Python script:

\begin{verbatim}
from sympy import *;

def uexact(x):
	return exp(sin(x)) * exp(cos(x));
	
def q1(x):
	return sin(x) * cos(x);
	
def q2(x):
	return -2 * cos(x)**4 + cos(x)**3 + (8 + 3 * sin(x)) * cos(x)**2 - (1 + sin(x)) * cos(x);

u = uexact(x);
ux = diff(u, x);
uxx = diff(ux, x);
uxxx = diff(uxx, x);
print trigsimp(simplify(uxxx + q1(x) * uxx + q2(x) * u(x) - 3 * exp(sin(x)) * exp(cos(x)))) == 0;
\end{verbatim}

By inspection, it should be clear that if uexact is an exact solution for the ode, then it ought to print out True; which is exactly what happens, and so the exactness condition is satisified.

\subparagraph{b}

Based on class discussion, pick 
\[u(x) = \sum \limits_{j=0}^N v_j \frac{\sin(\pi(x - x_j) / h)}{2 \pi / h \tan((x - x_j)/2)}\]

\subparagraph{c}


\paragraph{2}

Solve using chebfft.m + RK-4 integrator

\paragraph{3}

\subparagraph{a}

Assume that $\sigma(x) > 0$ for all $x \in [0,1]$, (otherwise the solution could be potentially imaginary or undetermined, and so it would be impossible to directly apply Sturm-Liouville theory).  From the boundary condition, we know that $\varphi(1) = 0$.  Combined with the asymptotic solution for the eigenmodes, we get the following equation for $\lambda$:
\[ 0 = \sigma(1)^{-\frac{1}{4}} \sin \left( \sqrt{ \lambda } \int \limits_0^1 \sqrt{ \sigma(\xi) } d \xi \right) \]
Which is true if and only if $\sin(...) = 0$, and so it must be that $... = 0, \pi, 2\pi, ... k \pi$ for all $k \in \mathbb{Z}$.  Thus:
\[ \sqrt{\lambda} \int \limits_0^1 \sqrt{\sigma(\xi)} d \xi = k \pi \]
And so
\[ \lambda = \left( \frac{k \pi}{ \int \limits_0^1 \sqrt{ \sigma(\xi) } d \xi } \right) ^2 \]

\subparagraph{b}

Once again, use chebfft.m + Rk-4 integrator.  Can recycle code from prog 2

\paragraph{4}

\subparagraph{a}

Just run the MATLAB script, add the necessary lines

\subparagraph{b}

Should be self-explanatory

\paragraph{5}

The forward Euler method is stable for values of $\Delta t \approx | \lambda |^{-1}$ where $\lambda$ is the largest eigenvalue of the system.  In this case, since we are dealing with a second-order dispersive system, the largest eigenvalue should be in $O(N^2)$ where $N$ is the number of samples.  So to ensure stability, the timestep should be in $\Delta t \in O(N^{-2})$.


For second part, just use formulas from text, plug into matlab

\end{document}

