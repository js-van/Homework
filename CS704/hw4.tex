\documentclass{article}

\usepackage{fullpage,amsmath,amsfonts}

\begin{document}

\newcommand{\hpl}[3]{ \left \{ #1 \right \} #2 \left \{ #3 \right \} }
\newcommand{\derive}[2]{\begin{array}{c} #1 \\ \hline #2 \\ \end{array} }
\renewcommand{\and}{,}

\paragraph{1}
To do this, we introduce the following preconditions and postconditions on the fragment, $C$
\[ \derive{}{\{ x = a \and y = b \} C \{ x = b \and y = a \}} \]
We now apply the assignment, consequence and composition axiom schemas several times to verify the expression:
\begin{equation*}
\begin{tiny}
\derive{
 \derive{
 	\derive{}{
 	  \hpl{x = a \and y = b}{x = x \oplus b}{x = a \oplus b \and y = b}} \and
 	\derive{}{
 	  \hpl{x = a \oplus b \and y = b}{y = x \oplus y}{x = a \oplus b \and y = a}}
 	}{
	\hpl{x = a \and y = b}{x = x \oplus y; y = x \oplus y ;}{x = a \oplus b \and y = a}} \and
 \derive{}{
 	\hpl{x = a \oplus b \and y = a}{x = y \oplus x ;}{x = b  \and y = a }}}{ 
 \hpl{x = a \and y = b}{x = x \oplus y; y = x \oplus y ; x = x \oplus y ;}{x = b \and y = a} }
\end{tiny}
\end{equation*}
(Note that $\oplus$ is used in place of $\wedge$ to denote exclusive-or to remove ambiguity.)

\paragraph{2}
Using the same pre/post conditions from problem 1, we proceed with the modified assignment axiom schema:
\begin{equation*}
\derive{}
{ \hpl{x = a \and y = b}{t = x; x = y; y = t;}{x = b \and y = a} }
\end{equation*}

Working from the left, we get the following:
\begin{equation*}
\derive{ \hpl{x = a \and y = b}{ t = x; }{ t = a \and x = a \and y = b } \and
\hpl{t = a \and x = a \and y = b}{ x = y ; y = t ; }{ x = b \and  y = a }
}
{ \hpl{x = a \and y = b}{t = x; x = y; y = t;}{x = b \and y = a} }
\end{equation*}

Taking the toprightmost expression, we further refine our derivation:
\begin{equation*}
\derive{
\hpl{t=a \and x=a \and y=b}{x = y;}{t = a \and x = b \and y = t} \and
\hpl{t = a \and x = b \and y = t}{y = t; }{x = b \and y = a}
}
{\hpl{t = a \and x = a \and y = b}{ x = y ; y = t ; }{ x = b \and  y = a }}
\end{equation*}

And to complete the derivation, we observe that:
\begin{equation*}
\derive{\hpl{t = a \and x = b \and y = t}{y = t; }{t = a \and x = b \and y = a}
\and (t = a \and y = a) \implies y = a
}
{\hpl{t = a \and x = b \and y = t}{y = t; }{x = b \and y = a}}
\end{equation*}

And so the segment correctly implements swap.

\paragraph{3}

To prove the correctness of this segment, we introduce the following loop invariant:
\begin{equation*}
\hpl{x = y * q + r \and 0 \leq r}{\text{while }r \geq y \text{ do } r = r - y ; q = q + 1 ; \text{ od}}{x = y q + r \and 0 \leq r \and r < y}
\end{equation*}

Applying the while axiom
\begin{equation*}
\derive{ 
\derive{
\derive{}
{\hpl{x = yq + r\and 0 \leq r \and r \geq y}{r = r - y;}{x = y \and (q + 1) + \and 0 \leq r}}
\and
\derive{}
{\hpl{x = y (q + 1) + r \and 0 \leq r \and r < y}{q = q + 1;}{x = y q + r \and 0 \leq r}}
}{\hpl{x = y q + r \and 0 \leq r \and r \geq y }{ r = r - y ; q = q + 1 ; }{x = y q + r \and 0 \leq r } } }
{ \hpl{x = y q + r \and 0 \leq r}{\text{while }r \geq y \text{ do } r = r - y ; q = q + 1 ; \text{ od}}{x = y q + r \and 0 \leq r \and r < y} }
\end{equation*}

Which we complete using the following:
\[ \derive{
(x = yq + r \implies x = y(q-1) + r-y) \and \hpl{x = y(q-1) + (r-y) \and 0 \leq r \and r \geq y}{r = r - y}{x = y(q-1) + r \and 0 \leq r \and r < y}
}
{\hpl{x = yq + r\and 0 \leq r \and r \geq y}{r = r - y;}{x = y \and (q + 1) + \and 0 \leq r}}
\]

To finish the proof, we must check the conditions on the intro fragment:
\[
\derive
{
	\derive{0 \leq x \and x = r \implies 0 \leq r}
	{\hpl{0 \leq x \and 0 < y \and r = x}
	{ r = x ; }
	{ x = r \and 0 \leq r}}
	\and
	\hpl{x = r \and 0 \leq r}
	{ q = 0; }
	{ x = yq + r \and 0 \leq r}
}
{
	\hpl{0 \leq x \and 0 < y}
	{r = x ; q = 0;}
	{x = yq + r \and 0 \leq r}
}
\]
And so the code correctly implements swap.


\end{document}

