\documentclass{article}

\usepackage{amsmath,amssymb,fullpage}

\begin{document}

\begin{flushright}
{\bf Mikola Lysenko

CS704, HW 1 

2-8-2010}
\end{flushright}

\paragraph{3.c}
Consider the $\lambda$-term:
\[ (\lambda x. \lambda y.y)( (\lambda x.xxx)(\lambda x.xxx) ) \]
If one performs the leftmost-outermost expansion then this expression reaches the normal form $\lambda y.y$ in one $\beta$-reduction.  Now define $S = \lambda x.xxx$ and consider any $\lambda$-term of the form:
\[ (\lambda x. \lambda y.y)( S^n ) \]
Where $n \geq 2$. Taking the rightmost outer most $\beta$-reduction leads to the following sequence:
\begin{eqnarray*}
& & ( \lambda x . \lambda y.y)( S^n ) \\
& \equiv & (\lambda x. \lambda y.y)(S^{n-2} (\lambda x.xxx) (\lambda x.xxx) ) \\
& \vdash_\beta & (\lambda x. \lambda y.y)(S^{n-2} (\lambda x.xxx) (\lambda x.xxx) (\lambda x.xxx) ) \\
& \equiv & ( \lambda x . \lambda y.y)( S^{n+1} )
\end{eqnarray*}
Thus each rightmost outermost $\beta$-reduction takes $(\lambda x. \lambda y.y)(S^n) \mapsto (\lambda x. \lambda y.y)(S^{n+1})$, and therefore will never reach the normal form.  So we conclude that the rightmost-outermost expansion rule does not always find a normal form if one exists.

\paragraph{5.a}
By the hypothesis, assume there exist lambda terms, $A \neq B$ and define 
\begin{eqnarray*}
f_1 & \equiv & \lambda s. \lambda t. s \\
f_2 & \equiv & \lambda s. \lambda t. t
\end{eqnarray*}
Now suppose that $S = K$.  If that were true, then it must be that:
\[ S f_2 f_1 A B = K f_2 f_1 A B \]
And that all of their reduced normal forms are equivalent.  Now reducing $S f_2 f_1 A B$ gives the following:
\begin{eqnarray*}
S f_2 f_1 A B & \equiv & (\lambda x . \lambda y . \lambda z . x z (y z) ) f_2 f_1 A B \\
& \vdash_\beta & (\lambda z . f_2 z (f_1 z)) A B \\
& \vdash_\beta & (f_2 A (f_1 A)) B \\
& \vdash_\beta & f_1 A B \\
& \vdash_\beta & A
\end{eqnarray*}
Similarly, reducing $K f_2 f_1 A B$ gives:
\begin{eqnarray*}
K f_2 f_1 A B & \equiv & ( \lambda x . \lambda y . x ) f_2 f_1 A B \\
& \vdash_\beta & f_2 A B \\
& \vdash_\beta & B
\end{eqnarray*}
But, this is a contradiction since $A \neq B$.  Therefore, it must be that $S \neq K$.

\paragraph{6}
\newcommand{\Plus}{\mathop{\underline{Plus}} \:}
\newcommand{\Times}{\mathop{\underline{Times}} \:}
\newcommand{\Comb}[1]{\mathop{\underline{#1}}}

We wish to construct a combinator, $\Plus$ such that for any Church numerals, $\Comb{a}, \Comb{b}$, with $a, b \in \mathbb{N}^+$:
\[ \Plus \Comb{a} \: \Comb{b} \to^*_{a,b} \Comb{a + b} \]
Pick:
\[ \Plus \equiv \lambda a . \lambda b . \lambda f . \lambda x . (a f) (b f x) \]
We now check the invariant on $\Plus$ by direct substitution:
\begin{eqnarray*}
\Plus \Comb{a} \: \Comb{b} & \equiv & (\lambda a . \lambda b . \lambda f . \lambda x . (a f) (b f x)) \Comb{a} \Comb{b}\\
& \vdash_\beta & \lambda f. \lambda x. (\Comb{a} f) (\Comb{b} f x) \\
& \vdash_\beta & \lambda f. \lambda x. (\lambda x'. f^a x') (f^b x) \\
& \vdash_\beta & \lambda f. \lambda x. f^a ( f^b x) \\
& \equiv & \lambda f. \lambda x. f^{a + b} x \\
& \equiv & \Comb{a + b}
\end{eqnarray*}
And so the definition of $\Plus$ satisifies the prescribed invariant.  Now for $\Times$, we wish to find a combinator which satisfies:
\[ \Times \Comb{a} \: \Comb{b} \to^*_{a,b} \Comb{a b} \]
Now we select:
\[ \Times \equiv \lambda a . \lambda b . \lambda f . \lambda x . (a ( b f )) x  \]
To check the invariant, we perform a similar expansion/$\beta$-reduction:
\begin{eqnarray*}
\Times \Comb{a} \: \Comb{b} & \equiv & (\lambda a . \lambda b . \lambda f . \lambda x . (a ( b f )) x) \Comb{a} \: \Comb{b} \\
& \vdash_\beta & \lambda f . \lambda x . (\Comb{a} ( \Comb{b} f )) x \\
& \vdash_\beta & \lambda f . \lambda x . (\Comb{b} f)^a x \\
& \vdash_\beta & \lambda f . \lambda x . (\lambda x' . f^b x')^a x \\
& \vdash_\beta & \lambda f. \lambda x . (\lambda x' . (f^b)^a x') x \\
& \vdash_\beta & \lambda f. \lambda x . (f^b)^a x \\
& \equiv & \lambda f. \lambda x. f^{a b} x \\
& \equiv & \Comb{a b}
\end{eqnarray*}
And so we conclude that $\Times$ is indeed a proper implementation of natural number multiplication.

\paragraph{10.a}
Consider the choice:
\[ W = ((\lambda w. \lambda n. n (w w) ) ( \lambda w. \lambda n (w w) )) \]
Then for any $\lambda$-term $N$ we have:
\begin{eqnarray*}
W N & = & ((\lambda w. \lambda n. n (w w) ) ( \lambda w. \lambda n (w w) )) N \\
& \vdash_\beta & (\lambda n . n ((\lambda w. \lambda n. n (w w) ) ( \lambda w. \lambda n (w w) ))) N \\
& \vdash_\beta & N ((\lambda w. \lambda n. n (w w) ) ( \lambda w. \lambda n (w w) )) \\
& \equiv & N W
\end{eqnarray*}
Thus, $W$ has the property that for all $N$:
\[ W N \to^*_{\alpha,\beta} N W \]

\paragraph{12.a}
If $\varphi$ is a fixed point combinator, then for all lambda terms $F$,
\[ \varphi F \to^*_{\alpha, \beta} F (\varphi F) \]
Which we check by expanding $\varphi F$:
\begin{eqnarray*}
\varphi F & \equiv & \theta^{17} F \\
& \vdash_\beta^* & (\lambda m . m ( \theta^{17} m)) F \\
& \vdash_\beta & F ( \theta^{17} F) \\
& \equiv & F ( \varphi F)
\end{eqnarray*}
And so $\varphi$ is a fixed-point combinator.

\paragraph{13.a}
We begin by expanding $G Y$:
\begin{eqnarray*}
G Y & \equiv & ( \lambda y . \lambda f . f ( y f ) ) Y \\
& \vdash_\beta & \lambda f . f ( Y f ) \\
& \vdash_{\alpha, \beta}^* & \lambda f . f ( (\lambda f'. ( \lambda x. f'(xx)) ( \lambda x. f'(xx))) f )  \\
& \vdash_\beta & \lambda f . f ( (\lambda x . f(xx)) ( \lambda x. f(xx)) )
\end{eqnarray*}
Likewise, starting from $Y$ we have:
\begin{eqnarray*}
Y & \equiv & \lambda f. (\lambda x. f(xx)) (\lambda x. f(xx)) \\
& \vdash_\beta & \lambda f . f ( (\lambda x. f(xx)) (\lambda x. f(xx)) )
\end{eqnarray*}
Thus we have:
\[ Y \rightarrow^*_{\alpha,\beta} \lambda f . f ( (\lambda x. f(xx)) (\lambda x. f(xx)) ) \leftarrow^*_{\alpha,\beta} G Y \]
And so $Y = GY$, which by the hypothesis shows that $Y$ is a fixed point combinator.

\paragraph{13.b}
If $M$ is a fixed point combinator, then for any $F$:
\begin{eqnarray*}
GMF & \equiv & (\lambda y . \lambda f . f(yf)) M F\\
 & \vdash_\beta & (\lambda f . f(M f)) F \\
 & \vdash_\beta & F (M F)
\end{eqnarray*}
Likewise, $M F = F (M F)$ (by the fact that $M$ is a fixed-point combinator), and so we have that $M F = G M F$ for all $F$, and thus $M = G M$.

Next, if $M = GM$, then for any $F$ once again we have:
\begin{eqnarray*}
G M F & \equiv & (\lambda y . \lambda f . f (yf) ) M F \\
 & \vdash_\beta^* & F (M F)
\end{eqnarray*}
Therefore, $M F = F (M F)$ and so $M$ is a fixed-point combinator.

In conclusion, $M = GM \Leftrightarrow M F = F M F$ for all $\lambda$-terms $F$.

\paragraph{16}
Consider the $\lambda$-term:
\[ P_0 =  \lambda z . (\lambda x . x x)( (\lambda y . y) z ) \]
Applying $\beta$-reduction to the left-sub expression gives:
\[ P_1 =  \lambda z . ((\lambda y . y) z) ((\lambda y . y) z) \]
Similarly right reduction results in:
\[ P_2 =  \lambda z . (\lambda x . x x) z \]
Yet, it would be impossible to go find a common object $P_1 \Rightarrow P_3 \Leftarrow P_2$, as both of the $\beta$ reductions have a common ancestor.  Thus the new definition for walk is not even weakly confluent.

\end{document}

