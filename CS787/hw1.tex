\documentclass{article}

\usepackage{clrscode}

\title{CS787 Homework 1}
\author{Mikola Lysenko}

\begin{document}

\maketitle{}

\paragraph{} \textbf{1.}
"Cuts in $(S, \mathcal{I})$ are precisely cycles in $(S, I)$"
\vspace{0.08in}

We start by showing the forward direction of the equivalence:  For any cut $C$ of $(S, \mathcal{I})$ and maximal element $I \in \mathcal{I}$, there exists some non-empty $c \subseteq C \cap I$.  By duality, $I^* = S - I$ is maximal in $(S,\mathcal{I}^*)$ and $c \cap I^* = \emptyset$, so $c \cup I^* \notin \mathcal{I}^*$.  Removing any element from $C$ would result in there being some $I'$ where $C \cap I' = \emptyset$, resulting in $C$ being independent in $\mathcal{I}^*$.  Therefore $C$ is a cycle of $(S,\mathcal{I}^*)$.

And now we finish our proof by showing the other direction: Given a cycle $C$ of $(S, \mathcal{I}^*)$, for every maximal element $I^* \in \mathcal{I}^*$ there exists some non-empty $c \subset C$ such that $C - c \subseteq I^*$.  Because $c \not \subset I^*$ and $I = S - I^*$, $c \subset I$ with $I$ maximal in $(S, \mathcal{I})$.  Because $C$ is minimal, removing any element would cause there to be some maximal set in $\mathcal{I}^*$ containing $C$, so there would have to be some other maximal set in $\mathcal{I}$ disjoint from $C$.  Therefore, $C$ must be a cut of $(S, \mathcal{I})$.

\vspace{0.15in}
\noindent "The Blue rule in $(S, \mathcal{I})$ is the same as the Red rule in $(S, \mathcal{I}^*)$ with the ordering of weights reversed"
\vspace{0.08in}

We show this by playing a word-game.  The Blue rule for $(S, \mathcal{I})$ is:

\vspace{0.05in} \noindent \texttt{\small Find a cut with no blue element, pick an uncolored element of the cut of minimum weight and color it blue. }

\vspace{0.05in} By the above result, this is the same as saying for $(S, \mathcal{I^*})$:

\vspace{0.05in} \noindent \texttt{\small Find a cycle with no blue element, pick an uncolored element of the cycle of minimum weight and color it blue. }

\vspace{0.05in} Switching the term blue to red, and swapping orders gives the rule:

\vspace{0.05in} \noindent \texttt{\small Find a cycle with no red element, pick an uncolored element of the cycle of maximum weight and color it red. }

\vspace{0.05in} Which is precisely the red Rule for $(S, \mathcal{I}^*)$.

\paragraph{} \textbf{2.}  
Suppose that $|A| \neq |B|$, then without loss of generality let $|A| < |B|$.  By definition there exists some $b \in B$ such that $A' = A \cup \{ b \} \in \mathcal{I}$.  However, this contradicts the statement that $A$ is maximal.  Therefore we must conclude that all maximal elements of $\mathcal{I}$ are the same cardinality.

\paragraph{} \textbf{3.}
We first prove that each matroid, $(S,\mathcal{I})$, is a set system with the prescribed properties.  Observe that for any $A \subset S$, $(S \cap A, \mathcal{I} \cap 2^A)$ is a matroid (by the second matroid axiom).  From Problem 2 we know that for any maximal $I,J \in \mathcal{I} \cap 2^A$, $|I| = |J|$.  Thus, the matroid $(S, \mathcal{I})$ satisfies the hypothesis.

To prove the opposite direction, we next show that some set system $(S, \mathcal{I})$ having the stated properties is a matroid.  The first two matroid axioms are trivially satisfied by the non-emptiness and downward closure of $\mathcal{I}$\footnote{Errata added from email}.  To prove the third axiom, let $I,J \in \mathcal{I}$ with $|I| < |J|$ and pick maximal elements $K_I, K_J \in \mathcal{I} \cap 2^{I \cup J}$ such that $I \subseteq K_I$ and $J \subseteq K_J$.  From the hypothesis, we know that $|K_I| = |K_J|$, $|K_J| \geq |J|$ and $|J| > |I|$, thus $|K_I| > |I|$.  Therefore there must exist some $x \in K_I - I$ such that $I \cup \{ x \} \in \mathcal{I}$.  Moreover $K_I - I \subseteq J$, so $x \in J$, proving that $(S, \mathcal{I})$ is a matroid.

\paragraph{} \textbf{4.}
Let $(S, \mathcal{I})$ be a matroid with $I \in \mathcal{I}$ maximal, $x \notin I$ and $I' = I \cup \{ x \}$.  By definition $I' \notin \mathcal{I}$, so $I'$ must contain at least one cycle.  Suppose that there exist two distinct cycles $C, C' \subset I'$ and pick some $y \in C - C'$ to construct $I'' = I' - \{ y \}$.  By the cycle interchange lemma (proved in class), $I''$ is maximal in $\mathcal{I}$.  However, $C' \subset I''$ which contradicts $C' \notin \mathcal{I}$, and therefore the cycle $C$ is unique.

\paragraph{} \textbf{5.}
Suppose $A,B \subseteq G$ are transitive reductions of $G$ and $A \neq B$.  Because $|A|=|B|$, there must exist some edge $x \in A - B$ such that the transitive closure of $G - x$ is the transitive closure of $G$ (because $B$ is a transitive reduction and $B-x = B$).  However, because $G$ is directed, acyclic and simple, if some edge $x=(a,b)$ is in the transitive reduction, then there can be no other path from vertex $a$ to $b$.  Therefore, removing $x$ from $G$ must changes the transitive closure of $G$ and we have a contradiction.

Because of the above result, we may examine each edge of $G$ is any order, removing only those which change the transitive closure of $G$.  The resulting minimal graph is then the unique transitive reduction of $G$:

\begin{codebox}
\Procname{$\proc{Transitive-Reduction}( G = (V,E) )$}
\li $G' \gets G$
\li \For \kw{each} $e \in E$
\li     \Do \If{$\proc{Transitive-Closure}(G' - e) = \proc{Transitive-Closure}(G)$}
\li		$\proc{Remove}(G',e)$
        \End
\li \Return $G'$
\end{codebox}

Because $\proc{Transitive-Closure}$ can be computed in polynomial time and the main loop is called exactly once per edge, the total running time of $\proc{Transitive-Reduction}$ is polynomial in $|G|$.

\paragraph{} \textbf{6.}
First, observe that within any circuit there are an even number of edges (one in and one out each time a vertex is visited), thus every vertex of an Eulerian circuit has even degree.  

Next, we prove the existence of an Eulerian circuit constructively for some graph $G=(V,E)$ with each vertex having even degree.  To do this, we first show that for any vertex $v \in V$, there exists some cycle containing $v$.  Initially, construct any path, $P$ starting at $v$ which does not contain any repeat edges.  If $P$ loops back to $v$ at some point, then take this smaller loop as our circuit and be done with it.  Otherwise, it must be possible to extend $P$ because each vertex of $G$ has even degree, so there must exist some other edge coming out from the head of $P$ which may be added to $P$.  The only case where this fails to hold is at $v$, but once $P$ has reached $v$, then we have constructed a circuit and we are done.  Next, observe that if there are two cycles $C,C'$ sharing a common vertex, then the concatenation $CC'$ is also a circuit.  Finally, we conclude our argument by observing that the removal of some circuit, $C$ from $G$, does not change the parity of the degrees of $G$, and moreover for any vertex of $C$ with edges in $G-C$ must be part of a circuit in within a connected component.  Because $G$ is connected this construction may be expanded until all edges of $G$ have been exhausted, thus $G$ must contain an Eulerian circuit.  Algorithmically this yields the following:

\begin{codebox}
\Procname{$\proc{Eulerian-Circuit}(v)$}
\li $R \gets []$
\li \While $\proc{Has-Edge}(v)$
\li		\Do
\li		$e \gets \proc{Get-Edge}(v)$
\li		$\proc{Remove-Edge}(e)$
\li		$R \gets \proc{Eulerian-Circuit}(\proc{Next}(e,v)) + e + R$
		\End
\li	\Return $R$
\end{codebox}

The running time of $\proc{Eulerian-Circuit}$ is dominated by the while-loop, which is executed only while $v$ has an edge.  Because each iteration removes one edge, the total running time is bounded by $O(|E|)$.

\end{document}
